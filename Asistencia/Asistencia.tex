\documentclass[10pt]{article}\usepackage[]{graphicx}\usepackage[]{xcolor}
% maxwidth is the original width if it is less than linewidth
% otherwise use linewidth (to make sure the graphics do not exceed the margin)
\makeatletter
\def\maxwidth{ %
  \ifdim\Gin@nat@width>\linewidth
    \linewidth
  \else
    \Gin@nat@width
  \fi
}
\makeatother

\definecolor{fgcolor}{rgb}{0.345, 0.345, 0.345}
\newcommand{\hlnum}[1]{\textcolor[rgb]{0.686,0.059,0.569}{#1}}%
\newcommand{\hlstr}[1]{\textcolor[rgb]{0.192,0.494,0.8}{#1}}%
\newcommand{\hlcom}[1]{\textcolor[rgb]{0.678,0.584,0.686}{\textit{#1}}}%
\newcommand{\hlopt}[1]{\textcolor[rgb]{0,0,0}{#1}}%
\newcommand{\hlstd}[1]{\textcolor[rgb]{0.345,0.345,0.345}{#1}}%
\newcommand{\hlkwa}[1]{\textcolor[rgb]{0.161,0.373,0.58}{\textbf{#1}}}%
\newcommand{\hlkwb}[1]{\textcolor[rgb]{0.69,0.353,0.396}{#1}}%
\newcommand{\hlkwc}[1]{\textcolor[rgb]{0.333,0.667,0.333}{#1}}%
\newcommand{\hlkwd}[1]{\textcolor[rgb]{0.737,0.353,0.396}{\textbf{#1}}}%
\let\hlipl\hlkwb

\usepackage{framed}
\makeatletter
\newenvironment{kframe}{%
 \def\at@end@of@kframe{}%
 \ifinner\ifhmode%
  \def\at@end@of@kframe{\end{minipage}}%
  \begin{minipage}{\columnwidth}%
 \fi\fi%
 \def\FrameCommand##1{\hskip\@totalleftmargin \hskip-\fboxsep
 \colorbox{shadecolor}{##1}\hskip-\fboxsep
     % There is no \\@totalrightmargin, so:
     \hskip-\linewidth \hskip-\@totalleftmargin \hskip\columnwidth}%
 \MakeFramed {\advance\hsize-\width
   \@totalleftmargin\z@ \linewidth\hsize
   \@setminipage}}%
 {\par\unskip\endMakeFramed%
 \at@end@of@kframe}
\makeatother

\definecolor{shadecolor}{rgb}{.97, .97, .97}
\definecolor{messagecolor}{rgb}{0, 0, 0}
\definecolor{warningcolor}{rgb}{1, 0, 1}
\definecolor{errorcolor}{rgb}{1, 0, 0}
\newenvironment{knitrout}{}{} % an empty environment to be redefined in TeX

\usepackage{alltt}
\usepackage[utf8]{inputenc}
\usepackage[spanish]{babel}
\usepackage[letterpaper, margin=1.5in, headheight=16pt]{geometry}
\usepackage{mathtools}
\usepackage{amsfonts}
\usepackage{amsmath}
\usepackage{amssymb}
\usepackage{amsthm}
\usepackage{graphicx}
\usepackage{enumitem}
\usepackage{caption}
\usepackage{subcaption}
\usepackage{titling}
\usepackage{float}
\usepackage{xcolor}
\usepackage{hyperref}
\definecolor{cimatred}{RGB}{122,51,69}
\hypersetup{
    colorlinks=true,
    linkcolor=cimatred,
    filecolor=cimatred,      
    urlcolor=cimatred,
    citecolor=cimatred
}
\urlstyle{same}
\newtheorem{theorem}{Teorema}
\newtheorem{corollary}{Corolario}[theorem]
\newtheorem{lemma}[theorem]{Lema}
\newtheorem{definition}{Definición}
\newtheorem{plain}{Proposición}
\newtheorem*{remark}{Observación}
\spanishdecimal{.}
\newcommand{\N}{\mathbb{N}}
\newcommand{\Z}{\mathbb{Z}}
\newcommand{\Q}{\mathbb{Q}}
\newcommand{\R}{\mathbb{R}}
\newcommand{\C}{\mathbb{C}}
\newcommand{\norm}[1]{\left\|#1\right\|}
\newcommand{\abs}[1]{\left|#1\right|}
\newcommand{\comp}[1]{#1^\mathrm{C}}
\newcommand{\mean}[1]{\mathbb{E}\left[#1\right]}
\newcommand{\cov}[1]{\mathrm{cov}\left[#1\right]}
\newcommand{\prob}[1]{\mathbb{P}\left(#1\right)}
\newcommand{\var}[1]{\mathrm{Var}\left(#1\right)}

\renewcommand{\baselinestretch}{1}
\DeclareMathOperator{\indic} {\textup{\large 1}}
\DeclareMathOperator{\Prob} {\mathbb P}
\DeclareMathOperator{\E} {\mathbb E}
\DeclareMathOperator{\V} {\mathbb V}
\DeclareMathOperator{\Normal} {\mathcal{N}}


\usepackage{fixdif}


\author{José Miguel Saavedra Aguilar}
\title{Ayudantías de Modelos Estadísticos}
\IfFileExists{upquote.sty}{\usepackage{upquote}}{}
\begin{document}

\pagestyle{plain}

\setlength{\parskip}{10pt}
\setlength{\parindent}{5pt}
\begin{minipage}{0.2\linewidth}
\vspace{-1cm}
\includegraphics[width=0.9\linewidth]{logoCIMAT11.png}
\end{minipage}
\begin{minipage}{0.7\linewidth}
\vspace{-1cm}
\noindent {\large \color{cimatred}\textbf{Centro de Investigación en Matemáticas, A.C.}}\\
\textbf{Inferencia Estadística}
\end{minipage}
\vspace{-5mm}
\begin{center}
\textbf{\large \thetitle}\\   %TITULO
\vspace{3mm}
\theauthor
\end{center}
\vspace{-5mm}
\rule{\linewidth}{0.1mm}
\begin{knitrout}
\definecolor{shadecolor}{rgb}{0.969, 0.969, 0.969}\color{fgcolor}\begin{kframe}


{\ttfamily\noindent\bfseries\color{errorcolor}{\#\# Error in read\_chunk("{}Asistencia.R"{}): no se pudo encontrar la función "{}read\_chunk"{}}}\end{kframe}
\end{knitrout}
\section{Introducción a Inferencia Estadística}

\subsection{Ejemplo 1}

Para una v.a. $X\sim \mathrm{exp}$, graficamos la función de densidad para distintas tasas $\lambda = 5, 2, 0.5$.








Ahora, graficamos las respectivas funciones de distribución de $X$.






Para $\lambda=0.5$, simulamos una muestra aleatoria de tamaño $n=100$ datos de $X$.




%Los datos que simulamos tienen media muestral $mean(z)$ y varianza muestral $var(z)$.\\
Ordenamos los puntos simulados en $x_{(1)},x_{(2)}, \ldots, x_{(n)}$. Les asociamos $k_i$ definido por
\begin{align}
k_i &= \frac{i}{n+1}
\end{align}



Para conocer más sobre la función de distribución empírica, pueden consultar \href{https://en.wikipedia.org/wiki/Empirical_distribution_function}{Wikipedia}.
\section{Ejemplos de Knitr}
Se ejemplifica el uso de knitr. Es recomendable consultar el libro de Yihui Xie \cite{Xie2015} para mayor información.
\subsection{Ejemplo 2}
Una variable aleatoria discreta $X$ tiene función de masa de probabilidad:
$$
\begin{array}{cccccc}{x} & {0} & {1} & {2} & {3} & {4} \\ \hline p(x) & {0.1} & {0.2} & {0.2} & {0.2} & {0.3}\end{array}
$$
Utilicen el teorema de la transformación inversa para generar una muestra aleatoria de tamaño 1000 de la distribución de $X$. Construyan una tabla de frecuencias relativas y comparen las probabilidades empíricas con las teóricas.\\
Repitan considerando la función de R sample.


\subsection{Ejemplo 3}
Obtengan una muestra de $10,000$ números de la siguiente distribución discreta:
$$
p(x)=\frac{2 x}{k(k+1)}, x=1,2, \ldots, k
$$
para $k=100$


\subsection{Ejemplo 4}
Una compañía de seguros tiene 1000 asegurados, cada uno de los cuales presentará de manera independiente una reclamación en el siguiente mes con probabilidad $p = 0.09245$. Suponiendo que las cantidades de los reclamos hechos son variables aleatorias Gamma(7000,1), hagan simulación para estimar la probabilidad de que la suma de los reclamos exceda $\$ 500,000$.


\section{Tarea 3}
\subsection{Ejercicio 1}
Simula las siguientes muestras Poisson, todas con $\lambda = 3$, pero de distintos tamaños, $n = 10,20,40,80,200$. Para cada muestra de estas tres calcula los tres estimadores de momentos dados en las notas en la pág. 3, $\lambda_1 , \lambda_2$ y $\lambda_3$ .


\subsection{Ejercicio 2}
Simula una muestra de $n = 15$ variables aleatorias independientes $X_1, ..., X_n$, idénticamente distribuidas como normales con media $\mu = 60$ y parámetro de escala$ \sigma = 5$.\\

Calcula los estimadores de momentos de $\mu$ y $\sigma$ basados en ecuaciones de los primeros dos momentos, los primeros no centrados y los segundos momentos centrados. Denota a estos estimadores como $\hat{\mu}$y $\hat{\sigma}$.

En una misma figura, grafica la función de distribución teórica con línea continua, la distribución estimada con guiones y la función de distribución empírica, graficando puntos de las siguientes coordenadas
$$\left( x_i ,  \frac{i}{n+1} \right)$$
para $i=1,...,n$.

Repite lo mismo pero ahora para $n=30$ y luego para $n=100$.





\section{Bootstrap no paramétrico}
Para esta ayudantía, nos basamos en el libro \cite{Wasserman2004}. Haremos bootstrap no paramétrico para estimar la desviación estándar de dos estimadores de la tasa $\lambda$ de una v.a. exponencial. Sea $X\sim \mathrm{exp}(\lambda = 5)$. Recordemos
\begin{align*}
  \mean{X} &= \frac{1}{\lambda}\\
  \var{X} &= \frac{1}{\lambda^2}
\end{align*}
Utilizaremos dos estimadores obtenidos por el método de momentos,
\begin{align*}
  \hat{\lambda} &= \frac{n}{\sum\limits_{i=1}^n X_i}\\
  \tilde{\lambda} &= \sqrt{\frac{n-1}{\sum\limits_{i=1}^n \left(X_i - \bar{X} \right)^2}}
\end{align*}
Iniciamos con una muestra aleatoria de $n=300$ v.a. exponenciales con $\lambda=5$.

Ahora, definimos las estadísticas que utilizaremos.

Para el bootstrap, tomaremos $k=100$ muestras de tamaño $m=30$ para cada una de las estadísticas $\hat{\lambda}$ y $\tilde{\lambda}$.

Finalmente, estimamos la desviación estándar de las estadísticas $\hat{\lambda}$ y $\tilde{\lambda}$.


\section{Bootstrap paramétrico}
Haremos bootstrap paramétrico para estimar la desviación estándar de dos estimadores de la tasa $\lambda$ de una v.a. exponencial. Sea $X\sim \mathrm{exp}(\lambda = 5)$. Recordemos
\begin{align*}
  \mean{X} &= \frac{1}{\lambda}
  \var{X} &= \frac{1}{\lambda^2}
\end{align*}
Utilizaremos dos estimadores obtenidos por el método de momentos,
\begin{align*}
  \hat{\lambda} &= \frac{n}{\sum\limits_{i=1}^n X_i}\\
  \tilde{\lambda} &= \sqrt{\frac{n-1}{\sum\limits_{i=1}^n \left(X_i - \bar{X} \right)^2}}
\end{align*}
Iniciamos con una muestra aleatoria de $n=300$ v.a. exponenciales con $\lambda=5$.

Ahora, definimos las estadísticas que utilizaremos.

Para el bootstrap paramétrico, tomaremos $k=100$ muestras de tamaño $m=30$ para cada una de las estadísticas $\hat{\lambda}$ y $\tilde{\lambda}$, considerando la tasa $\lambda=\hat{\lambda}$.

Finalmente, estimamos la desviación estándar de las estadísticas $\hat{\lambda}$ y $\tilde{\lambda}$.


\section{Bootstrap paramétrico con intervalos de confianza}
Haremos bootstrap paramétrico para estimar la desviación estándar del cuantil muestral 0.975 de una muestra normal. Sea $X\sim \mathrm{Normal}(\mu = 0.2, \sigma = 0.2)$.

Nuestro estimador será el cuantil muestral 0.975:


Para el bootstrap paramétrico, tomaremos $k=100$ muestras de tamaño $m=30$ para cada el cuantil muestral 0.975, considerando los EMVs de una Normal:
\begin{align*}
  \hat{\mu} &= \frac{\sum\limits_{i=1}^n X_i}{n}\\
  \hat{\sigma} &= \sqrt{\frac{\sum\limits_{i=1}^n \left(X_i - \bar{X} \right)^2}{n-1}}
\end{align*}

Finalmente, estimamos la desviación estándar del cuantil muestral 0.975. Además, tomaremos intervalos del $96\%$ de confianza considerando el teorema del límite central:
\begin{knitrout}
\definecolor{shadecolor}{rgb}{0.969, 0.969, 0.969}\color{fgcolor}\begin{kframe}
\begin{alltt}
\hlstd{S1} \hlkwb{<-} \hlkwd{quantile}\hlstd{(meanEst1,} \hlnum{0.02}\hlstd{)}
\end{alltt}


{\ttfamily\noindent\bfseries\color{errorcolor}{\#\# Error in eval(expr, envir, enclos): objeto 'meanEst1' no encontrado}}\begin{alltt}
\hlstd{S2} \hlkwb{<-} \hlkwd{quantile}\hlstd{(meanEst1,} \hlnum{0.98}\hlstd{)}
\end{alltt}


{\ttfamily\noindent\bfseries\color{errorcolor}{\#\# Error in eval(expr, envir, enclos): objeto 'meanEst1' no encontrado}}\begin{alltt}
\hlstd{T1} \hlkwb{<-} \hlkwd{mean}\hlstd{(meanEst1)} \hlopt{+} \hlkwd{sd}\hlstd{(meanEst1)} \hlopt{/} \hlkwd{sqrt}\hlstd{(k)} \hlopt{*} \hlkwd{qnorm}\hlstd{(}\hlnum{0.02}\hlstd{, k} \hlopt{-} \hlnum{1}\hlstd{)}
\end{alltt}


{\ttfamily\noindent\bfseries\color{errorcolor}{\#\# Error in eval(expr, envir, enclos): objeto 'meanEst1' no encontrado}}\begin{alltt}
\hlstd{T2} \hlkwb{<-} \hlkwd{mean}\hlstd{(meanEst1)} \hlopt{+} \hlkwd{sd}\hlstd{(meanEst1)} \hlopt{/} \hlkwd{sqrt}\hlstd{(k)} \hlopt{*} \hlkwd{qnorm}\hlstd{(}\hlnum{0.98}\hlstd{, k} \hlopt{-} \hlnum{1}\hlstd{)}
\end{alltt}


{\ttfamily\noindent\bfseries\color{errorcolor}{\#\# Error in eval(expr, envir, enclos): objeto 'meanEst1' no encontrado}}\begin{alltt}
\hlkwd{print}\hlstd{(}\hlkwd{sprintf}\hlstd{(}\hlstr{"El cuantil empírico 0.02 es %2.3f"}\hlstd{, S1))}
\end{alltt}


{\ttfamily\noindent\bfseries\color{errorcolor}{\#\# Error in eval(expr, envir, enclos): objeto 'S1' no encontrado}}\begin{alltt}
\hlkwd{print}\hlstd{(}\hlkwd{sprintf}\hlstd{(}\hlstr{"El cuantil empírico 0.98 es %2.3f"}\hlstd{, S2))}
\end{alltt}


{\ttfamily\noindent\bfseries\color{errorcolor}{\#\# Error in eval(expr, envir, enclos): objeto 'S2' no encontrado}}\begin{alltt}
\hlkwd{print}\hlstd{(}\hlkwd{sprintf}\hlstd{(}\hlstr{"El cuantil T 0.02 es %2.3f"}\hlstd{, T1))}
\end{alltt}


{\ttfamily\noindent\bfseries\color{errorcolor}{\#\# Error in eval(expr, envir, enclos): objeto 'T1' no encontrado}}\begin{alltt}
\hlkwd{print}\hlstd{(}\hlkwd{sprintf}\hlstd{(}\hlstr{"El cuantil T 0.98 es %2.3f"}\hlstd{, T2))}
\end{alltt}


{\ttfamily\noindent\bfseries\color{errorcolor}{\#\# Error in eval(expr, envir, enclos): objeto 'T2' no encontrado}}\end{kframe}
\end{knitrout}

\bibliography{biblio}
\bibliographystyle{IEEEtranS}
\end{document}
